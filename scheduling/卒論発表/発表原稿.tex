\documentclass[a4j]{jarticle}
\title{\huge 卒研の原稿}
\author{大原 源悠}
\date{\today}
\renewcommand{\baselinestretch}{0.9}
\setlength{\textwidth}{16.92cm}
\setlength{\textheight}{24.6cm}
\setlength{\oddsidemargin}{-0.50 cm}
\setlength{\evensidemargin}{-0.50 cm}
%\setlength{\oddsidemargin}{0 cm}
\setlength{\topmargin}{-1.5cm}%{-0.8 cm}
\setlength{\abovedisplayskip}{-2.0cm}
\setlength{\belowdisplayskip}{-2.0cm}
\usepackage[dvipdfmx]{graphicx}

\begin{document}
%
\maketitle
%

\begin{itemize}
\item スライド$1$
\end{itemize}
システム最適化研究室の都$14-33$の大原源悠です.
本日はよろしくお願いします.
私の研究テーマは「関心度の高い他研究室の発表聴講可能性を高める特別研究報告審査会スケジュールの作成」です.
まず初めに本研究の背景と目的について説明していきたいと思います.
%

\begin{itemize}
\item スライド$2$
\end{itemize}
本研究の背景についてですが,本学科では,毎年$2$月に特別研究報告審査会が行われます.その特別研究報告審査会のスケジュールは,毎年教員が手動で作成していました.
このスケジュール作成には,満たすべき要件が複数あり,作成に多くの手間を要してしました.

そこで,若林がスケジュール作成を最適化問題として定式化し,スケジュール一覧表を自動で作成するインターフェースを作成しました.
しかし,この研究で作成されたスケジュールに対して,ある教員から「研究内容が近い研究室の教員が,お互いの研究室の発表を聞けるようにしたい」という要望を頂きました.
また,現在のインターフェースについていくつか修正すべき事項がありました.

次に、本研究の目的について説明します。

本研究の目的は$2$つあり、$1$つは昨年度作成されたスケジュール作成問題の改良で,もう$1$つは、現在のインターフェースの利便性向上です.
この$2$つの目的について順に説明していきたいと思います。
%

\begin{itemize}
\item スライド$3$
\end{itemize}
まず,スケジュール作成問題の改良について説明します.

本研究では,頂いた要望を実現するために,新たな制約条件を追加した$2$種類の最適化モデルを作成しました.
頂いた要望の内容は,「研究内容が近い研究室の教員が,お互いの研究室の発表を聞けるようにしたい」という内容でした.
この要望を実現するために,「関心度が高く発表を聞きに行きたい研究室同士の発表セッションが重複しない」という制約条件を追加したモデルを作成しました.

この制約条件について説明します.


\begin{itemize}
\item スライド$4$
\end{itemize}
例として,研究室$A$の教員がが研究室$B$の学生の発表を聞きに行きたい場合を考えます.

図のように,研究室$A$と研究室$B$の発表セッションが重複している場合,研究室$A$に所属する学生と,研究室$B$に所属する学生の発表順番が重複してしまう可能性があります.この場合,研究室$A$の教員$A$は,研究室$B$の学生の発表を聞きに行くことができません.

そこで,スケジュールを図のように,研究室$A$と研究室$B$の発表セッションをずらしたスケジュールに変更すると,教員$A$は,自分の担当する学生が発表するセッション終了後,研究室$B$が発表を行うセッションに参加することが可能になります.

本研究では,この制約条件をこのように定式化しました.

また,他にも,...や...のような条件があります.

\begin{itemize}
\item スライド$5$
\end{itemize}
次に,インターフェースの利便性向上について説明します.

昨年度,各研究室に研究室データを記入したExcelファイルを提出して頂きました.
しかし,研究室によってこのExcelファイルへの記入方法が異なっていました.
具体的には,教員名や氏名,学籍番号などの書き方が異なっていました.

そのため,データの一部を手動で修正する必要がありました.

%

\begin{itemize}
\item スライド$6$
\end{itemize}
そこで,本研究では,このExcelファイルへの記入方法を統一するように改良しました.
具体的には,図のように,入力項目をドロップダウンメニューから選択するようにしました.
また,学生の氏名や研究室IDは,学籍番号や研究室名をメニューから選択すると,自動で入力されるように改良しました.
本研究ではこのような改良を行いインターフェースの利便性を向上しました.
%
\\

\begin{itemize}
\item スライド$7$
\end{itemize}
次に,本研究で作成した,$2017$年度の特別研究方向審査会スケジュールについて説明します.
今年度のスケジュールは,このような条件の下,作成しました.

\begin{itemize}
\item スライド$8$
\end{itemize}
実際に作成したスケジュールを確認すると,図のようになりました.
この図は,関心度が高く発表セッションが重複しないことが望ましい研究室の組合せを同じ色で表しています.

図を見ると,同じ色の研究室同士の発表セッションは重複しておらず,それぞれ互い違いになっていることが確認できます.
このことから,本研究で新たに追加した,制約条件が正しく作用していることが確認できました.



\begin{itemize}
\item スライド$9$
\end{itemize}
最後にこれまでのまとめと今後の課題について説明します。
本研究では,昨年度のスケジュール作成問題を改良しました.具体的には,新たな制約条件を追加した$2$種類の最適化モデルを作成し,$2017$年度の特別研究報告審査会スケジュールを作成しました.
また,研究室データを記入するExcelファイルの記入方法を統一し,インターフェースの利便性を向上しました.

今後の課題として,制約条件の定式化に間違いがあり,一部スケジュールを手動で修正する必要があったことが挙げられます.
また,今後,「より最適なスケジュール」を作成するために,制約条件の追加・変更が必要になると考えられます.

以上で発表を終わります.ご清聴ありがとうございました.
%

\end{document}

