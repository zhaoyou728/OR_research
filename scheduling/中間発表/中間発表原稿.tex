\documentclass[a4j]{jarticle}
\title{\huge 中間発表の原稿}
\author{大原 源悠}
\date{\today}
\renewcommand{\baselinestretch}{0.9}
\setlength{\textwidth}{16.92cm}
\setlength{\textheight}{24.6cm}
\setlength{\oddsidemargin}{-0.50 cm}
\setlength{\evensidemargin}{-0.50 cm}
%\setlength{\oddsidemargin}{0 cm}
\setlength{\topmargin}{-1.5cm}%{-0.8 cm}
\setlength{\abovedisplayskip}{-2.0cm}
\setlength{\belowdisplayskip}{-2.0cm}
\usepackage[dvipdfmx]{graphicx}

\begin{document}
%
\maketitle
%

\begin{itemize}
\item スライド$1$
\end{itemize}
システム最適化研究室の都$14-33$の大原源悠です。
本日はよろしくお願いします。
私の研究テーマは、特別研究報告審査会の柔軟なスケジュール作成とインタフェースの利便性向上です。
まず初めに本研究の背景と目的について説明していきたいと思います。
%

\begin{itemize}
\item スライド$2$
\end{itemize}
本研究の背景についてですが、本学科では、毎年$2$月に特別研究報告審査会が行われます。その特別研究報告審査会のスケジュールは、毎年教員が手動で作成していました。
このスケジュール作成には、満たすべき要件が複数あり、作成に多くの手間を要してしました。
そこで、若林がスケジュール作成を最適化問題として定式化し、スケジュール一覧表を自動で作成するインターフェースを作成しました。
次に、本研究の目的についついて説明します。\\
本研究の目的は$2$つあり、$1$つは特別研究報告審査会のより柔軟なスケジュールを作成する事で、もう$1$つは、現在のインターフェースの利便性を向上させることです。
この$2$つの目的について順に説明していきたいと思います。
%

\begin{itemize}
\item スライド$3$
\end{itemize}
まず、特別研究報告審査会の概要について説明します。
特別研究報告審査会は$2$日間に渡って$3$部屋で行われます。
各教室において、$1$日目の午前、$1$日目の午後、$2$日目の午前でそれぞれ$2$セッション実施します。
よって、$1$部屋につき計$6$セッション、$3$部屋で合計$18$セッションの実施となります。
学生はこの$18$セッションのどこかで必ず発表します。
%

\begin{itemize}
\item スライド$4$
\end{itemize}
次に若林が作成した最適化モデルの内容について説明します。
スケジュールが満たすべき要件には、絶対制約と考慮制約の二種類があります。
絶対制約は、「学生は、自分自身と単教員がともに参加可能なセッションで発表する」や、「研究室が同じ学生は教室をまたいで同時刻のセッションで発表しない」などの必ず満たさなければならない制約のことです。
そして、考慮制約は、「同時刻に行われるセッションの発表人数の最大と最小の差は$1$以下とするのが望ましい」や「各研究室はすべての時間帯で発表するのが望ましい」などの、できるだけ満たしたい制約のことです。
このような制約を最適化問題として定式化し、最適化モデルを作成しました。

\begin{itemize}
\item スライド$5$
\end{itemize}
次に、現在の最適化モデルの問題点のついて説明します。
現在の最適化モデルでは、求解時間が長く、最適解を求め切れていないケースが存在します。
具体的には、各セッションの発表人数の上限に奇数が増えると、図のように求解時間が急激に伸びる傾向にあります。
求解時間が$10800$秒、つまり$3$時間を超えると、その時点での暫定的な解を最適解とし、スケジュールを作成しています。
これでは、本当に最適なスケジュールを作成できていない可能性があるため、最適化モデルの再検討が必要です。
%

\begin{itemize}
\item スライド$6$
\end{itemize}
また、スケジュール作成について、ある教員から追加したい要件があるという要望を頂いています。
その追加したい要件は、「研究内容が近い研究室の教員が、お互いの研究室の発表を聞けるようにしたい」という内容です。
現在のモデルでは、発表順序を考慮しておらず、発表日程、教室、セッションのみを考慮しています。
そのため、次の例のようなケースが発生する可能性があります。
%
\\

\begin{itemize}
\item スライド$7$
\end{itemize}
例として、研究室Aの教員が、研究室Bの学生の発表を聞きたい場合を考えます。
現在のモデルでは、図のように、研究室Aの学生の発表順序と、研究室Bの学生の発表順序が同じになる場合があります。
このような時、教員Aは研究室Bの学生の発表を聞きに行くことができません。
\\
しかし、発表順序を考慮すると、図のように、研究室Aと研究室Bの学生の発表順序をずらすことが可能になります。
よって、教員Aは、自分の担当する学生の発表が終わり次第、教室を移動し、研究室Bの学生の発表を聞きに行くことができます。
本研究では、このような、より柔軟なスケジュール作成を可能にする最適化モデルの作成を目指します。
また、今後アンケートを実施し、スケジュール作成について、追加・変更したい部分の調査を行う予定です。
以上が、追加したい要件の内容です。
%


\begin{itemize}
\item スライド$8$
\end{itemize}
次に、現在のインターフェースの機能について説明します。
まず、インターフェースに、利用教室や、各セッションでの発表人数の上限、一体運用を行っている研究室の組み合わせなどの情報を入力し、インターフェースを実行します。
%

\begin{itemize}
\item スライド$9$
\end{itemize}
すると、図のような、各研究室に記入して提出してもらった情報を元に、最適化計算用のデータファイルが自動で作成されます。
そして、作成したデータファイルと問題を定式化したモデルファイルを使い、最適化ソルバを用いて最適化問題を解きます。
最後に求解結果から、発表者の情報を各セッションごとに整理し、スケジュール一覧表をPDFファイルで作成します。
%

\begin{itemize}
\item スライド$10$
\end{itemize}
次に、現在のインターフェースの問題点について説明します。
現在のインターフェースは、モデルファイルやバッチファイルなど、インターフェースを実行する前に多くの準備が必要になります。
また、インターフェースの利用環境が変わると、図のようなバッチファイルや、絶対パスの変更などの設定が必要になります。
これらの操作は手間を要し、また、インターフェースの保守性が損なわれる可能性があるため、改良が必要です。
%


\begin{itemize}
\item スライド$11$
\end{itemize}
最後にこれまでのまとめと今後の課題について説明します。
これまでに、現在の最適化モデルと、インタフェースの問題点について説明しました。
最適化モデルについては、求解時間の短縮と、追加したい要件の実現が必要です。、
インタフェースについては、全体的な利便性の向上が必要です。
今後の課題としては、最適化モデルを再検討します。
また、アンケートを実施し、追加・変更したい部分の調査も行う予定です。
最後に、Excel以外でのインターフェースを作成し、利便性の向上を目指します。
以上で私の中間発表を終わらせていただきます。
ご清聴ありがとうございました。
%

\end{document}

