\documentclass[14pt,dvipdfmx]{beamer}

\usetheme{Madrid}
\setbeamertemplate{navigation symbols}{}

\usepackage{graphicx}
\usepackage{amsmath}
\usepackage{amsfonts}
\usepackage{amssymb}
\usepackage{comment}
\usepackage{txfonts}
\usepackage{ascmac}

%\mathversion{bold}
\renewcommand{\familydefault}{\sfdefault}
\renewcommand{\kanjifamilydefault}{\gtdefault}
\setbeamerfont{title}{size=\large,series=\bfseries}
\setbeamerfont{frametitle}{size=\large,series=\bfseries}
\setbeamertemplate{frametitle}[default][center]
\usefonttheme{professionalfonts}
%
\newcommand{\bd}[1]{\mbox{\boldmath $#1$}}
\def\smskip{\par\vskip 5pt}
\def\QED{\hfill $\Box$ \smskip}
%
\title[第$4$章 文の推敲]{第$4$章 文の推敲}
\author[都$14-33$大原源悠]{都$14-33$ 大原源悠 \\ システム最適化研究室}
\institute[システム最適化研究室]{}
\date{December $8$, $2017$}

\begin{document}
%
\frame{\titlepage}
%

\frame{
  \frametitle{この章で学ぶこと}
  %
  文を書くにあたって,
  \begin{itemize}
  \item どのような文が誤解されやすいか
  \item どう書き直せば誤解されなくなるか
  \end{itemize}
  %
  という点について学ぶ
}

\frame{
  \frametitle{誤解される文とは ($1/2$) }
  %
  \begin{itembox}[l]{リスと私}
   私はカメラを抱えたまま寄ってきたリスにクルミをあげた.
  \end{itembox}
  この文章はどのような情景を描いているだろうか
    
}

\frame{
  \frametitle{誤解される文章とは ($2/2$) }
  %
  $2$つの読み方ができる
  \begin{itemize}
  \item カメラを抱えているのは私
    \begin{itemize}
    \item リスが寄ってきたので,私はカメラを抱えたままで\\
      クルミをあげた
    \end{itemize}
    \end{itemize}
    \begin{itemize}
    \item カメラを抱えていたのはリス
      \begin{itemize}
      \item 高度な知能を持ったリスが寄ってきたので,\\
        私はクルミをあげた
      \end{itemize}
    \end{itemize}
    このような誤解を受けないように注意する点が\\いくつかある
  %
}

\frame{
  \frametitle{短くする}
  %
  \begin{itembox}[l]{悪い例}
    私はカメラを抱えたまま寄ってきたリスにクルミをあげた.
  \end{itembox}
  悪い点:複数の動詞が一つの文で使われている
  %
  \begin{itembox}[l]{改善例$1$:読点を打つ}
    私はカメラを抱えたまま,寄ってきたリスにクルミをあげた.
  \end{itembox}
  \begin{itembox}[l]{改善例$2$:語順を変える}
    寄ってきたリスに,私はカメラを抱えたままクルミをあげた.
  \end{itembox}
}

\frame{
  \frametitle{「の」の数に注意($1/2$) }
  %
  \begin{itembox}[l]{悪い例:「の」が多すぎる}
   反応の速度の測定の結果のデータの処理のための\\プログラムが必要です.
  \end{itembox}
  
  悪い点:「の」が多すぎて読みにくい
  
  \begin{itembox}[l]{改善例:「の」を減らす}
    反応速度の測定結果を処理するプログラムが\\必要です.
  \end{itembox}
  %
}

\frame{
  \frametitle{「の」の数に注意($2/2$)}
  %
  「の」を減らす方法
    \begin{itemize}
    \item 単純に「の」を減らす
      \begin{itemize}
      \item 「反応の速度」→「反応速度」
      \end{itemize}
      
    \item 冗長な部分を省く
      \begin{itemize}
      \item 「処理のための」→「処理の」
      \end{itemize}
      
    \item 「の」を別の語に変える
      \begin{itemize}
      \item 「処理のプログラム」→「処理するプログラム」
      \end{itemize}
    \end{itemize}
  %
}

\frame{
  \frametitle{明確にする}
  %
  文を明確にし,誤解を生まないようにする
  \begin{itembox}[l]{ 悪い例:複数の対応関係が不明確}
    Table$1$とTable$2$に示した加算と乗算の表を\\見てください.
  \end{itembox}

  悪い点:対応関係が不明確
  \begin{itembox}[l]{改善例:複数の対応関係を明確にした}
    加算の表(Table$1$)と乗算の表(Table$2$)を\\見てください.
  \end{itembox}
  複数の対応関係を明確にすることで誤解を\\生まなくなる
  %
}

\frame{
  \frametitle{言外の意味($1/2$)}
  %
  \begin{beamerboxesrounded}[shadow=false]
    {言外の意味とは}
    文章を読んでいると「文字としては書かれていないが,自然と心に浮かんでくる意味」がある.\\
    これを
    \begin{bfseries}
      言外の意味
    \end{bfseries}
    という.
  \end{beamerboxesrounded}
  %
  \begin{itembox}[l]{例}
    薬品 A は,薬品 B とは反応しない
  \end{itembox}
  「薬品 B とは」とわざわざ書かれているため,\\「 A と B 以外の薬品が存在するのかも」と感じる
}

\frame{
  \frametitle{言外の意味($2/2$)}
  %
  \begin{itembox}[l]{改善例}
    薬品 A は,薬品 B とは反応しない.それは薬品 B に含まれていいる成分が……だからである.\\
    このように,薬品 A は薬品 B とは反応しないが,\\薬品 C とは反応する.なぜなら,薬品 C は薬品 B とは違い,……だからである.
  \end{itembox}
  読者に「疑問」や「不満」を感じさせないようにする必要がある
  %
}

\frame{
  \frametitle{二重否定を避ける}
  %
  二重否定は,誤解を生む可能性がある
  \begin{itembox}[l]{二重否定の例}
    この公式で解が求められない$2$次方程式はない.
  \end{itembox}
  %
  二重否定を避けることで,文がすっきりする
  \begin{itembox}[l]{改善例}
    どんな$2$次方程式の解も,この公式で求められる.
  \end{itembox}
  %
  さまざまな書き方が可能なので,前後の文脈や文章の流れを考えつつ書き換えてみることが重要
}

\frame{
  \frametitle{この章のまとめ}
  %
  読者に誤解を与えないために注意する点がある
  \begin{itemize}
  \item 文を短くする
  \item 「の」の数に注意する
  \item 明確な文を書く
  \item 言外の意味に注意する
  \item 二重否定に注意する
  \end{itemize}
  %
  文を推敲するときには
  \begin{bfseries}
      わざと意地悪な読み方をする
  \end{bfseries}
  ことが大切.
  \\誤解する可能性を少しでも減らした文にすることを\\心掛けよう.
}

\end{document}

%%%%% End of file %%%%%
