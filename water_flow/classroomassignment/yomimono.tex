\documentclass{jsarticle}
\usepackage{amsmath}
\usepackage[dvipdfmx]{graphicx}
\usepackage{array}

\begin{document}
\begin{itemize}
\item 1ページ\\
システムモデリング研究室の岡崎俊介です.
よろしくお願いします.
テーマは「最大の移動時間を最小化する教室割当問題の定式化と求解」です.
この課題は去年の卒業生である鈴木健太さんの論文の継続課題です.
まず,教室割当問題について説明します.\\

\item 2ページ\\
教室割当とは,各授業をどの教室で開講するか,その割当を決めることです.
そして教室割当問題とは,授業間の教室移動時間が最小となる教室割当を求める問題のことを指します.
先行研究では,移動時間の総和を最小化するような教室割当を求める問題を,最適化問題として定式化し,求解しています.

先行研究では,第4学舎で開講される授業に対する教室割当について,全学期,全曜日,午前・午後の24通りについて計算しています.
しかし,計算時間がその上限である1時間を超えてしまうことが多く,厳密解が得られたものが,6通りしかありませんでした.
そこで先行研究の問題点を分析しました.
すると,まず,目的関数が複雑な構造であることがわかりました.
また,用意された制約式に誤りがある,さらに制約と変数の数が非常に多いことがわかりました.
これらの問題点により,計算に時間がかかっているのではないかと推測しました.

そこで,本研究では,先行研究とは異なる視点から教室割当問題の求解を行い,また制約式の改良を行うことで最適解を求めようと考えています.


\item 3ページ\\
本研究では,最適解を求めるために,まず,教室割当問題の目標を変更しました.
先行研究では,移動時間の総和を最小化することで,最適解を求めようとしています.
しかし,先ほど述べたように,このために目的関数・制約条件が複雑になっています.
そこで本研究では,目標を少し変え,教室割当問題を,授業間の最大の移動時間が最小となる教室割当を求める問題とし,求解することとしました.
つまり,本研究では,一番長い移動時間をできるだけ短くすることによって,全体の移動時間も同時に抑えることができるのではないか,と考えています.


\item 4ページ\\
ここで,制約条件の定式化の例を紹介します.
これは,本研究における制約の1つである,「各授業には必ず1つの教室を割り当てなければならない」を定式化したものです.
この式は,$p$限目に開講される授業$j$には,教室$i$が必ず割り当てられることを示しています.
また,目的関数は,制約がそれぞれ違反したとき,値が大きくなり,それを最小化しています.
ここでは,時間の都合上,その他の制約式の詳細について説明することができません.
回覧中の卒業論文を参照してください.

\item 5ページ\\
次に,本研究の実験手順を説明します.
本研究では,最大移動時間の上限を定めるパラメータ$r$を設け,これを調整しながら,繰り返し問題を解き,最小の$r$を最適解として求めています.
ここではその手順について簡単に説明します.
本研究では,はじめは$r$を300秒としています.
実験を行い,実行可能解を得ることができれば,次に,0秒との中間値である150秒で実験します.
同様に75秒を実験し,解が求まらなければ,75秒と150秒の間の値を実験します.
このように二分探索することで,パラメータ$r$の最小値を求めます.


\item 6ページ\\
ここから,本研究での実験結果を説明します.
まず,全24通りのデータに対して,最大の移動時間の最小値を求める実験を行いました.
この表は,春学期での実験結果です.
このように,春学期の全てのデータで最適解を得ることができました.
また,秋学期についても同様に,全てのデータで最適解を得ることができています.

\item 7ページ\\
例えば,春学期月曜午前での実験結果は,最大移動時間は96秒,平均移動時間は58.3秒でした.
また,この解を得る際の計算時間は,4.34秒でした.
ただし,この計算時間はパラメータ$r$が96秒のときの計算時間です.
この図は,移動時間ごとの学生数を示した図です.
このように,全体的に移動時間を短くできていることが分かります.

\item 8ページ\\
本研究では,全24通りで解を求めることができました.
また,全ての解において,計算時間は0.13秒から80.62秒の間で,平均の計算時間は10.42秒でした.
この表は,本研究と先行研究の,春学期月曜午前での実験結果を比較したものです.
このように,先行研究では最適解を求めることができませんでしたが,本研究では短い計算時間で求めることができました.
また,制約と変数の数は,大幅に少なくすることができました.

\item 9ページ\\
さらに本研究で提案したモデルでは,開講する教室に希望がある場合,それを考慮することができるようになっています.
3つの授業に対して,希望する教室を設定して解いた結果がこちらです.
赤が希望をしていない場合,青が希望をした場合です.
計算時間をそれほど変えずに,最適解を求めることができました.

\item 10ページ\\
まとめとしまして,本研究では,先行研究の誤りを正し,また異なる視点から教室割当問題を定式化し,求解しました.
その結果,制約と変数の数を大幅に少なくでき,最大の移動時間を最小化する教室割当問題を,短い計算時間で解くことができました.
また,希望を考慮する教室割当も行うことができました.

今後の課題として,最大でない移動時間についてはまだ最小化する余地があること,また,本研究での教室割当を実際に使用するためのインターフェイスの開発が挙げられます.
以上で発表をおわります,ありがとうございました.


\if0


\item 9ページ\\
実験1では,春学期月曜午後と,木曜午後での最適解が,他の曜限での解よりも大きくなっていました.
しかし,この表にように,どちらも平均移動時間は最大移動時間と大きな差があることがわかります.
これは,特定の授業に対する移動時間だけが大きくなっているからではないか,と考えました.
そこで,本研究では実験2として,この2つの曜限において,最も移動時間がかかっている授業を取り除いて再度計算を行いました.

\item 10ページ\\
この表は,実験2での結果です.
この表で分かるように,実験2では,どちらの曜限も最大移動時間を大幅に小さくすることができました.
また,平均移動時間も小さくなっていることがわかります.
この結果は,移動時間が大きくなりそうな授業の教室割当をあらかじめ決めておくことで,残りの授業に対するより良い教室割当を得ることができるということを示しています.

\item 11ページ\\
これは,実験1と実験2での,移動時間ごとの学生数を比較したグラフです.
赤が実験1の結果で,青が実験2の結果を示しています.
このように,このグラフからも,実験2ではより学生の移動時間が小さくなっていることが分かります.

\fi


\if0
先行研究の制約式の改良,追加を行いました.
また,目的関数の設定の変更などを行っています.
これらにより,目標した教室割当問題を,全24通りで,素早く解を求めることを目標としています.

\item 4ページ\\
次に先行研究の制約条件を説明します.\\
まず絶対制約です.絶対制約とは,守らなければいけない制約で,
「1 つの曜限における各教室には授業は1つしか割り当てられない」など4つあります\\
次に考慮制約です.考慮制約とは,できるだけ守りたい制約のことで,
「移動時間は指定した時間以内であることが好ましい」など3つあります.\\
また,目的関数は,移動を行う全学生の移動時間の総和と,考慮制約を満たせていない場合に発生する違反点数の合計を最小化した値としています.\\
本研究では,検証の結果,先行研究の考慮制約に改良すべき点を発見しました.



\item 5ページ\\
これから先行研究から改良したことについて説明します.\\
まず,考慮制約1の「特別連続授業は同じ教室で開講されることが好ましい.」についてです.特別連続授業とは,微分積分と微分積分演習のような関連性の高い,連続で開講される2 つの授業の事を指します,\\
$u_{i,j}$は教室$i$で授業$j$が行われるかを表す01変数です.なのでこの式は,特別連続授業である授業$j_1,j_2$が別の教室で行われるならば,違反の指標を立てるという式になっています.\\
特別連続授業はほとんどの学生が連続受講しています.なので本研究では,この制約を絶対制約とすることで,混雑発生の可能性をなくし,実行可能領域を狭めています.
この式は,特別連続授業である授業$j_1,j_2$は,どちらも教室$i$で開講されるということを示しています.\\

\item 6ページ\\
次に,考慮制約2の「移動時間は指定した時間以内であることが好ましい」についてです.
この式は,検証した結果,正しい制約式になっていないことがわかりました.
なので本研究では,この制約式を正し,またこの制約も絶対制約にすることで,一定時間以上の休み時間を確保しました.
この式は,教室$i_1,i_2 $間で移動する生徒がいて,その移動時間が一定時間を越える場合,教室$i_1$で授業$j_1$,教室$i_2$で授業$j_2$の両方が割り当てられることはないということを示しています.\\

\item 7ページ・8ページ\\
最後に,考慮制約3の「教室内の人が入れ替わる際の混雑が発生しないほうが好ましい」についてです.この制約は,教室を出入りする人数が一定数を超えると違反指標を立てるという制約になっています.この制約を定式化すると,非常に長くなり説明に時間がかかってしまうので,式は割愛します.\\
先行研究では,混雑は授業終了直後に起こると考えていてます.なので混雑の原因となる人数は,教室から出て行く人,前の時限に授業がなく,次にこの教室で受講している人,そして前の時限に隣の教室で授業を受けてから次にこの教室で受講している人,の3つであるとなっています.対して本研究では,遠方の教室でも早めに授業が終了することなども考慮し,休み時間内に教室を出入りするすべての人を混雑の対象とすることにしました.\\

\item 9ページ\\
次に,追加する制約条件について説明します.
例えば学生や教員には,自分の研究室に近いところなど,授業を開講したい教室の希望があると思います.しかし教室希望を絶対制約にしてしまうと,教員の希望がかぶった場合など,問題が発生してしまいます.なので本研究では,これを考慮制約としました.$I_j$は,希望する教室の集合です.この式は各授業において希望した教室以外で授業があったとき,目的関数に違反点数が加算される式になっています.\\

\item 10ページ\\
まとめとしまして,本研究ではまず先行研究を勉強し,理解するということをしました.
そして,制約条件から問題点を発見し,改良・追加を行いました.\\
今後の研究予定は,このようなことを行おうと考えています.
以上で発表を終わります.ありがとうございました.
\fi
\end{itemize}
\end{document}
