\documentclass[a4j]{jarticle}
\renewcommand{\baselinestretch}{0.9}
\usepackage{url}
\setlength{\textwidth}{16.92cm}
\setlength{\textheight}{24.6cm}
\setlength{\oddsidemargin}{-0.50 cm}
\setlength{\evensidemargin}{-0.50 cm}
\setlength{\topmargin}{-1.5cm}
\setlength{\abovedisplayskip}{-2.0cm}
\setlength{\belowdisplayskip}{-2.0cm}
\setlength{\columnseprule}{0.3pt}

\title{武田さんの卒論}
\author{都13-15 馬谷慎太郎}
\date{\today}
\begin{document}
\maketitle
\section{はじめに}
近年異常気象による集中豪雨が増加傾向にある.気象問題に伴って都市問題も深刻化している.都市部への人口密度やインフラの老朽化である.そのような条件下で想定外力「を超えた内水氾濫が発生するのは必至の状況である.それに適応するために,水防法が改訂されハード・ソフトの両面から浸水対策が掲げられた.なかでも地下空間の脆弱性は近年注目されている.福岡水害では時間雨量78.5mmの降雨が発生し,博多駅地下街に洪水が流れ込み,深刻な被害が生じた.被害特徴は地下内の電気整備室に流れ込み事業継続が困難になったことである.
\section{武田さん研究の目的}
\begin{itemize}
\item 降雨強度の変化がリードタイムと地下への流入量にどのような影響を与えるか
\item 止水板設置途中で浸水が始まる可能性が高い点を考慮し,シミュレーションを行い,止水板設置順序ごとに流入量がどのように変化するのか検討
\item 止水板を設置するタイミングの決定を重要視する.その決定方法によってリードタイムがどのように変わるか検討する.
 \item それに焦点を置いて,地下街マネジメントについて検討し,現在と今後の浸水対策について検討する.
  \end{itemize}
\section{武田さんの研究の特徴}
外力を短時間集中豪雨とする.そのため,地下街管理者は十分な対策がとれず,人員を十分確保できないリスクが生じ,地下街の止水板を全て設置できない.止水板の設置途中で浸水が始まる可能性が高い点を考慮し,止水板設置順序毎に流入量がどのように変化するのか検討する.その際止水板設置を開始するタイミングが重要になる.正しい情報が伝達される可能性も低く,適切なタイミングで設置できるとは限らない.決定方法によってリードタイムと流入量がどのように変化するのか検討する.また,対象外力が異なる3つのパターンの外力とし,降雨強度の変化がリードタイムと地下街への流入量にどのような影響を与えるのか検討する.
\section{対象地域と対象降雨}
対象地域は大規模地下空間を含む大阪市北区海老江処理地区内である.この地域は元々海面との標高差がなく,過去にもたびたび河川からの氾濫の被害に遭っていた.また,森兼らは同じ海老江処理地区を対象として,浸水しやすい地域の特徴を4つ挙げた.
\begin{itemize}
\item 流端の管渠が多数位置している
\item 大きな管径から小さな管径に変化している
\item 周辺よりも地盤高が低い
\item 処理場の排水区ではなく,中継ポンプの排水区に位置している
\end{itemize}
このような浸水しやすい地域を対象としている.
\subsection{分析対象地下空間}
本地下空間は主に3ヶ所に分かれている.また,浸水被害から施設全体を守り維持管理をする上では電気整備室の位置が重要となる.その位置と地上出入り口,防災拠点である防災センターの位置関係について図3-5に示す.本研究では止水板の効果について検討するが,従来の研究の結果より流入可能性がある入口の止水板のみを考える.
\subsection{浸水防止計画}
梅田地下空間浸水防止計画では,現況の浸水対策と今後の課題とする浸水対策が記載されている.
まず,地上入り口にある止水板の検証を行う.この止水板は全出入り口にあるわけではないのが現状である.さらに既往研究より流入割合が高い出入り口には止水板がない.これらの入り口の師丑番が設置が急務である.本研究では,全ての入り口に止水板がある状態を仮定し,止水板の設置順序によって流入量がどのように変化するのか検討する.
\section{検討内容及びケース}
\subsection{地下街が直面している問題}
本研究では,内水氾濫を対象としたシミュレーションを行う.そのため,突発的な短時間集中豪雨を対象降雨とする.突発的なため,地下街管理者は対応行動をとりにくい.また,降雨開始時刻の頻度を表すヒストグラフを見ると集中豪雨の開始時刻の頻度は夕刻~深夜に集中している.この時間帯は帰宅ラッシュの時間で人が集中し,地下街管理者も不足することが予想される.
よって時間内に全ての止水板を設置することは難しく,設置途中で浸水が始まる可能性が高い.また,正しい情報が伝達される可能性も低く,適切なタイミングで止水板を設置できるとは限らない.そのような状況下で重要となる止水板の設置順序と設置開始のタイミング決定方法を説明する
\subsection{警備員行動ルール}
梅田地下空間浸水防止計画に基づき,配置人数を2人1組で6組とする.人の方向スピードを時速4kmとする.また,梅田に設置されている止水板には様々な種類の止水板があり,種類によって設置方法も変わってくる.
\begin{itemize}
\item ハンドル起上式\ 3分
\item 横引き式\ 3分
\item 観音開き\ 3分
\item 改良型組み立て式\ 5分
\item 組み立て式\ 5分
\item パネル差込み式\ 3分
  \end{itemize}
\section{外力とFTAの関係}
\subsection{FTA}
本研究では降雨強度の異なる3パターンの降雨を外力とする.降雨強度が異なると地下空間流入時刻が変わり,地下街管理者の対応行動も変わる.図4.4
\subsection{FTA}
\begin{itemize}
\item 情報の重要性
  \item ヒューマンエラー削減のための機器の自動化や緊急訓練の実施
  \end{itemize}
このFTAと問題設定に基づき,本シミュレーションで検討する項目を決定する.人間のミスを全く考慮しないシミュレーションではなく,解析により人間行動の分析を行うのが本研究の特徴である.FTAで挙げられた情報の認識方法,人間が行う止水板設置行動を分析する.
\subsection{止水板設置順序}
止水板の設置順序は3つである.
\begin{itemize}
\item 既往シミュレーション結果に基づき,流入量が多い出入り口から順に設置
\item 防止センター付近の止水板から順に設置
  \item 地下内電気整備室に影響を与える止水板から順に設置
  \end{itemize}
\section{内水氾濫シミュレーション}
分布型モデルを用いて解析をおこなった.以下の図より説明.
\subsection{止水板設置順序毎の流入比較}
本研究では流入開始が始まった出入り口には設置しないと考えている
\end{document}
