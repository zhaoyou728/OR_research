 %%%%%%%%%%%%%%%%%%%%%%%%%%%%%%%%%%%%%%%%%%%%%%%%%%
% 
% [title]
%                                    Hiroshige DAN
%                                      yyyy/mm/dd
%%%%%%%%%%%%%%%%%%%%%%%%%%%%%%%%%%%%%%%%%%%%%%%%%%

\documentclass{jarticle}
\usepackage{latexsym}
%\usepackage[fleqn]{amsmath}
\usepackage{amsmath}
\usepackage{amsfonts}

\setlength{\textheight}{43\baselineskip}
\setlength{\textwidth}{47zw}
\setlength{\oddsidemargin}{10pt}
\setlength{\evensidemargin}{10pt}
\setlength{\topmargin}{0pt}

\newtheorem{proposition}{Proposition}[section]
\newtheorem{theorem}{Theorem}[section]
\newtheorem{lemma}{Lemma}[section]
\newtheorem{corollary}{Corollary}[section]
\newtheorem{definition}{Definition}[section]
\newtheorem{remark}{Remark}[section]
\newtheorem{example}{Example}[section]
\newtheorem{algorithm}{Algorithm}[section]
\newtheorem{motivation}{Motivation}[section]
\newtheorem{assumption}{Assumption}[section]

\def\smskip{\par\vskip 5 pt}
\def\QED{\hfill $\Box$ \smskip}
\newcommand{\bd}[1]{\mbox{\boldmath $#1$}}

\makeatletter
\renewcommand{\theequation}{\thesection.\arabic{equation}}
\@addtoreset{equation}{section}
\makeatother

%%%%%%%%%%%%%%%%%%%%%%%%%%%%%%%%%%%%%%%%%%%%%%%%%%
\begin{document}

%%%%%%%%%%%%%%%%%%%%%%%%%%%%%%%%%%%%%%%%%%%%%%%%%%
\section{最適化モデル}

本節では,本研究で提案する最適化モデルについて説明する.

\subsection{モデル中で利用する記号}

ここでは,提案する最適化モデル中で利用する記号について整理する.

\begin{itemize}
\item 集合・添字
      \begin{itemize}
    \item $t \in T$: 時刻を表す添字 $t$ とその集合 $T$
    \item $s \in S$: シナリオを表す添字 $s$ とその集合 $S$
      \end{itemize}
\item パラメータ
      \begin{itemize}
    \item $\tilde{s}_{ts}$: 時刻 $t$ におけるシナリオ $s$ に対応する,
          時刻 $t - 1$ におけるシナリオ番号(@@@ 図)
    \item $r_t$: 時刻 $t$ における売電価格
    \item $P_{ts}$: 時刻 $t$ におけるシナリオ $s$ の発生確率
    \item $w_{ts}$: 時刻 $t$ におけるシナリオ $s$ での発電量
    \item $CA$: 蓄電池容量
    \item $\tilde{P}_U, \tilde{P}_L$: 風力発電設備の運用が破綻する確
          率の許容値.$\tilde{P}_U$ は,蓄電量が,蓄電池容量の
          $70$\% を上回るシナリオの発生確率の許容値.$\tilde{P}_L$
          は $30$\% を下回るシナリオの発生確率の許容値.
    \item $\tilde{C}_U, \tilde{C}_L$: 最終時刻における蓄電量の期待値
          の上下限.
    \item $M$: 非常に大きな正の定数(いわゆる big-M として利用する)
      \end{itemize}
\item 変数
      \begin{itemize}
    \item $WB_t$: 時刻 $t$ において,発電機(発電器?)で発電した電
          気のうち,蓄電池に蓄電される電気量の期待値
    \item $WG_t$: 時刻 $t$ において,発電機(発電器?)で発電した電
          気のうち,電力購入会社へ売電される電気量の期待値
    \item $BG_t$: 時刻 $t$ において,蓄電池にある電気のうち,電力購入
          会社へ売電される電気量の期待値
          \begin{itemize}
          \item $WG_t + BG_t$: 時刻 $t$ において電力購入会社へ売電さ
            れる電気量の期待値(= 電力購入会社への売電通告量)
          \end{itemize}
    \item $C_{ts}$: 時刻 $t$ におけるシナリオ $s$ においての蓄電量
    \item $\delta^{1U}_{ts}, \delta^{1L}_{ts}, \delta^{2U}_{ts},
          \delta^{2L}_{ts}$: 中間変数($0\mathchar`-1$ 変数)
    \item $v^U_{ts}, v^L_{ts}$: 中間変数
      \end{itemize}
\end{itemize}

@@@「風力発電設備の運用が破綻する」という状態について,ここより前に記載し
ておく必要がある.運用が破綻した状態とは,蓄電池での蓄電量が,蓄電池の容
量(?)に対して一定の範囲内に収まってないことをいう.ここでいう「一定の範囲」
は施設によって定められるが,$30$\% から $70$\% 程度の範囲を指すことが多い
(@@@ ほんとか?).

@@@ $n$ 期であることを明確に

\subsection{目的関数}

ここでは,提案するモデルの目的関数について説明する.

本モデルの目的は,(\ref{subsec:constraints} 節で示す制約条件の下で)売電
価格の合計を最大化することである.売電価格は,次式のように書くことができ
る:
%
\begin{eqnarray}
\sum_{t = 1}^n r_t \cdot (WG_t + BG_t)
  \label{eq:obj}
\end{eqnarray}
%
(@@@ あとで若干の修正をするかも)

\subsection{制約条件}
\label{subsec:constraints}

ここでは,提案するモデルの制約条件について説明する.

提案するモデルでは,大きく分けて次の…

\begin{eqnarray}
WB_t + WG_t = \sum_{s \in S} P_{ts} w_{ts}
  \label{eq:con1}
\end{eqnarray}

\begin{eqnarray}
&& - M (1 - \delta^{1L}_{ts})
  \le 0.98 (WG_t + BG_t) - w_{ts}
  \le M \delta^{1L}_{ts}
  \label{eq:con2} \\
&& v^L_{ts} - M (1 - \delta^{1L}_{ts})
  \le 0.98 (WG_t + BG_t) - w_{ts}
  \le v^L_{ts} + M (1 - \delta^{1L}_{ts})
  \label{eq:con3} \\
&& - M \delta^{1L}_{ts} \le v^L_{ts} \le M \delta^{1L}_{ts}
  \label{eq:con4}
\end{eqnarray}

\begin{eqnarray}
&& - M (1 - \delta^{1U}_{ts})
  \le w_{ts} - 1.02 (WG_t + BG_t)
  \le M \delta^{1U}_{ts}
  \label{eq:con5} \\
&& v^U_{ts} - M (1 - \delta^{1U}_{ts})
  \le w_{ts} - 1.02 (WG_t + BG_t)
  \le v^U_{ts} + M (1 - \delta^{1U}_{ts})
  \label{eq:con6} \\
&& - M \delta^{1U}_{ts} \le v^U_{ts} \le M \delta^{1U}_{ts}
  \label{eq:con7}
\end{eqnarray}

\begin{eqnarray}
C_{ts} = C_{t-1,\tilde{s}(s)} - \frac{1}{0.9} v^L_{ts} + 0.95 v^U_{ts}
  - Const
  \label{eq:con8}
\end{eqnarray}

\begin{eqnarray}
&& - M (1 - \delta^{2L}_{ts})
  \le 0.3 CA - C_{ts}
  \le M \delta^{2L}_{ts}
  \label{eq:con9} \\
&& - M (1 - \delta^{2U}_{ts})
  \le C_{ts} - 0.7 CA
  \le M \delta^{2U}_{ts}
  \label{eq:con10}
\end{eqnarray}

\begin{eqnarray}
&& \sum_{s \in S} \delta^{2U}_{ts} P_{ts}
  \le \tilde{P}_U
  \label{eq:con11} \\
&& \sum_{s \in S} \delta^{2L}_{ts} P_{ts}
  \le \tilde{P}_L
  \label{eq:con12}
\end{eqnarray}

\begin{eqnarray}
\tilde{C}_L
  \le \sum_{s \in S} C_{ns}
  \le \tilde{C}_U
  \label{eq:con13}
\end{eqnarray}

\subsection{解くべき最適化問題}

%%%%%%%%%%%%%%%%%%%%%%%%%%%%%%%%%%%%%%%%%%%%%%%%%%%
\section*{Acknowledgment}

%%%%%%%%%%%%%%%%%%%%%%%%%%%%%%%%%%%%%%%%%%%%%%%%%%%

\begin{thebibliography}{99}
\bibitem{BT95}
P. T. Boggs and J. W. Tolle,
Sequential Quadratic Programming,
{\it Acta Numerica}, (1995), 1-51.
\end{thebibliography}
\end{document}

%%%%%%%%%%%%%%%%%%%%%%%%%%%%%%%%%%%%%%%%%%%%%%%%%%%

%%%%% End of file %%%%%