\documentclass[11pt,a4paper]{jsarticle}
%
\usepackage{amsmath,amssymb}
\usepackage{bm}
\usepackage[dvipdfm]{graphicx}
\usepackage{ascmac}
\usepackage{float}
%
\setlength{\textwidth}{\fullwidth}
\setlength{\textheight}{39\baselineskip}
\addtolength{\textheight}{\topskip}
\setlength{\voffset}{-0.5in}
\setlength{\headsep}{0.3in}
%
\newcommand{\divergence}{\mathrm{div}\,}  %ダイバージェンス
\newcommand{\grad}{\mathrm{grad}\,}  %グラディエント
\newcommand{\rot}{\mathrm{rot}\,}  %ローテーション
%
\pagestyle{myheadings}
\markright{}
\begin{document}
%
%
\section*{前年度の運用計画のモデル}
\begin{flushleft}
【昼間の運用計画のモデル】

蓄電池に十分充電ができており、夜間に充電した電力を売りたいとき

(T > bc(100 - s)/100[kw])
\end{flushleft}

\begin{flushleft}
   *送電量:102X(1 - (q/100))/100

   *充電量:X(1 - (q/100))(-2 + p)/100
\end{flushleft}

\begin{flushleft}
蓄電池の容量が維持充電量となったとき(T = bc(100 - s)/100[kw])
\end{flushleft}

\begin{flushleft}
  - (p < -2)の時

   *送電量:102X(1 - (q/100))/100

   *充電量:X(1 - (q/100))(-2 + p)/100
\end{flushleft}

\begin{flushleft}
  - (-2 \hspace{0.3em}\raisebox{0.4ex}{$<$}\hspace{-0.75em}\raisebox{-.7ex}{=}\hspace{0.3em} p \hspace{0.3em}\raisebox{0.4ex}{$<$}\hspace{-0.75em}\raisebox{-.7ex}{=}\hspace{0.3em} 2)の時

   *送電量:X(1 - (q/100))(1 + (p/100)

   *充電量:0
\end{flushleft}

\begin{flushleft}
  - (p > 2)の時

   *送電量:102X(1 - (q/100))/100

   *充電量:X(1 - (q/100))(2 + p)/100
\end{flushleft}


\begin{flushleft}
【夜間の運用計画のモデル】

蓄電池の容量がマージン以下で充電したいとき

(T = bc × (100 - r)/100[kw])
\end{flushleft}

\begin{flushleft}
  - (p < 2)の時

   *送電量:102X(1 - (q/100))/100

   *充電量:X(1 - (q/100))(-2 + p)/100
\end{flushleft}

\begin{flushleft}
  - (-2 \hspace{0.3em}\raisebox{0.4ex}{$<$}\hspace{-0.75em}\raisebox{-.7ex}{=}\hspace{0.3em} p \hspace{0.3em}\raisebox{0.4ex}{$<$}\hspace{-0.75em}\raisebox{-.7ex}{=}\hspace{0.3em} 2)の時

   *送電量:X(1 - (q/100))(1 + (p/100))

   *充電量:0
\end{flushleft}

\begin{flushleft}
  - (p > 2)の時

   *送電量:102X(1 - (q/100))/100

   *充電量:X(1 - (q/100))(-2 + p)/100

\end{flushleft}


%
%
\end{document}