\documentclass[11pt,a4paper]{jsarticle}
%
\usepackage{amsmath,amssymb}
\usepackage{bm}
\usepackage[dvipdfm]{graphicx}
\usepackage{ascmac}
\usepackage{float}
%
\setlength{\textwidth}{\fullwidth}
\setlength{\textheight}{39\baselineskip}
\addtolength{\textheight}{\topskip}
\setlength{\voffset}{-0.5in}
\setlength{\headsep}{0.3in}
%
\newcommand{\divergence}{\mathrm{div}\,}  %ダイバージェンス
\newcommand{\grad}{\mathrm{grad}\,}  %グラディエント
\newcommand{\rot}{\mathrm{rot}\,}  %ローテーション
%
\title{シナリオ・ツリー型モデルを利用した風力発電設備の安定運用計画の立案}
\author{シス05-19 臼木 誠}
\date{\empty}
\begin{document}
\maketitle

% 目次の表示
\tableofcontents

% 本文
\section{はじめに}

\section{風力発電}

\subsection{仕組みとメリット・デメリット}

\subsubsection{WF(ウィンドファーム)の発電、送電}

\subsubsection{風速による発電量}

\subsubsection{風車の種類}

\subsubsection{メリット・デメリット}

\subsection{制御型}

\subsubsection{出力一定制御型}

\subsubsection{出力変動緩和型}

\subsection{蓄電池}

\subsubsection{蓄電池を設置する理由}

\subsubsection{種類(鉛蓄電池、NAS電池)}

\subsection{風況予測技術}

\subsection{日本と世界の導入状況、事業}

\section{研究目的}

\subsection{風力発電設備の安定運用}

\subsection{前年度の研究とその問題点}

\section{使用する手法と提案するモデル}

\subsection{数理計画モデル(シナリオ・ツリー)}

\subsection{モデル化に必要な要素}

\subsection{運用計画のモデル化}

\section{数値実験}

\subsection{実験説明}

\subsection{実験内容}

\subsection{実験結果、考察}

\section{おわりに}

\section{謝辞}

\section{参考文献}



%







